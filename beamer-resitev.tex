\documentclass{beamer}

\usepackage[utf8]{inputenc}
\usepackage[T1]{fontenc}
\usepackage[slovene]{babel}
\usepackage{lmodern}

\usetheme{frankfurt}
\usecolortheme{seahorse}

\beamertemplatenavigationsymbolsempty

% pisava
\usepackage{palatino}
\usefonttheme{serif}

\begin{document}

% ===================================================================

\title{Računanje števila $\pi$}
\author{Cene Novak}

%====================================================================
\begin{frame}
   \maketitle
\end{frame}

%====================================================================

\begin{frame}
    \frametitle{Verjetnostni način}
    \begin{block}{Približek verjetnosti}
        Pri metu igle na črtast list papirja beležimo uspele in neuspele poskuse in izračunamo približek verjetnosti
        $$ P(A) \approx \frac{\text{število uspešnih poskusov}}{\text{št. vseh poskusov}}. $$
    \end{block} \pause

    Približek števila $\pi$ izračunamo s pomočjo igle dolžine~$l$ in razdalje med črtami~$d$ tako
    $$\frac{m}{n} \approx P = \frac{2l}{\pi d} \quad \Rightarrow \quad \pi \approx \frac{2l}{d}\frac{n}{m}. $$
\end{frame}

%====================================================================

\begin{frame}
    \frametitle{Računanje približkov}
    Vsi našteti so uporabili razdaljo med črtama $d=1$.
    \begin{table}
        \centering
    \begin{tabular}{|l|c|c|c|l|} 
        \hline
         & \textbf{igla} & \textbf{št. metov} & \textbf{št. uspešnih} & \textbf{približek} \\ 
       \hline
       Wolf & 0,8 & 5000 & 2532 & 3,1596\\ 
       \hline
       Lazzarini & 0,83 & 3408 & 1808 & 3,1415929 \\
       \hline
       Gridgeman & 0,7857 & 2 & 1 & 3,143\\
       \hline
       \end{tabular}
    \end{table}
\end{frame}

%===================================================================

\begin{frame}
    \frametitle{Ostali načini}
    Različni avtorji so različno obravnavali število $\pi$:
    \begin{itemize}
        \item $\displaystyle{\frac{\pi}{2} = \frac{2 \cdot 2 \cdot 4 \cdot 4 \cdot 6\cdot 6 \cdots}{1\cdot 3 \cdot 3 \cdot 5 \cdot 5 \cdot 7 \cdots}}$
        \item $\displaystyle{\frac{\pi^2}{6} = 1 + \frac{1}{2^2} + \frac{1}{3^2} + \frac{1}{4^2} + \cdots}$
        \item $\displaystyle{\frac{1}{\pi} = \sum_{n=0}^{\infty}{2n \choose n}^3 \frac{42n+5}{2^{12n+4}}}$      
    \end{itemize}

    Več o tem si lahko preberemo \href{http://www.nieuwarchief.nl/serie5/pdf/naw5-2000-01-3-254.pdf}{tukaj}.
\end{frame}

\end{document}
