\documentclass{article}
\usepackage[utf8]{inputenc}
\usepackage[T1]{fontenc}
\usepackage[slovene]{babel}


\usepackage{lmodern}

\usepackage{amsmath}
\usepackage{amssymb}
\usepackage{amsthm}
\usepackage{graphicx}

{\theoremstyle{definition}
\newtheorem{definicija}{Definicija}
\newtheorem{zgled}[definicija]{Zgled}
}

\setlength{\parindent}{0mm}


\title{Buffonova igla}
\author{Cene Novak}

%==================================================
\begin{document}
\maketitle

\begin{abstract}
    Nekaj o verjetnosti in Buffonovem problemu iz~\cite{book}
\end{abstract}

\section{Verjetnost dogodka}

\emph{Verjetnost dogodka} je število, ki pove, kolikšna je možnost, da se zgodi nek dogodek.

\begin{definicija}
Verjetnost dogodka $A$ je razmerje med številom ugodnih izidov $n$
in vseh možnih izidov $m$ in se izračuna na sledeč način:
\begin{equation}
\label{eqn:verjetnost}
P(A) = \frac{m}{n} = \frac{\text{št. ugodnih izidov}}{\text{št. vseh možnih izidov}}.
\end{equation}
\end{definicija}


\section{Buffonov problem}
Problem Buffonove igle je zastavil in rešil francoski 
matematik G.~L.~Leclerc v 18.~stoletju. Problem se glasi:
``Na črtast list, kjer so črte razmaknjene za $d$
vržemo iglo dolžine $l$. Kolikšna je verjetnost, da se igla dotika kakšne črte?''


\subsection{Rešitev za kratko iglo}
%================================================================= uredi
Na problem pogledamo na sledeč način $P(\{\text{igla seka črto}\}) = P(\{ l\cos{\theta} > H\})$ 
in si pomagamo z grafom~\ref{fig:graf}.

\begin{figure}[!h]
\centering
\caption{Graf za pomoč}
\label{fig:graf}
\includegraphics[width=4cm]{graf}
\end{figure}

Ploščina pod krivuljo predstavlja ugodne izide. Vsi možni izidi pa 
ustrezajo ploščini pravokotnika. Uporabimo~\eqref{eqn:verjetnost} 
na strani \pageref{eqn:verjetnost} in računamo

\begin{align*}
    P(\{\text{igla seka črto}\}) 
    &= \frac{\displaystyle{\int_{0}^{\frac{\pi}{2}} l\cos{\theta}\,d\theta}}{\frac{\pi}{2}d} \\
    &= \frac{ l (\sin{\frac{\pi}{2}} - \sin{0})}{\frac{\pi}{2}d}\\
     &= \frac{2 l}{\pi d}.
\end{align*}

    
\subsection{Rešitev za daljšo iglo}
Za daljšo iglo ($ l \geq d$) le navedimo rezultat
\begin{equation}
    \label{eqn:daljsa}
    P(\{\text{igla seka črto}\}) = 
    \frac{2}{\pi} \left(\arccos\left(\frac{d}{l}\right) + \frac{l - \sqrt{ l^2 - d^2}}{d} \right).
\end{equation}

Če združimo, dobimo
$$
   P(A) = \begin{cases}
    \frac{2 l}{\pi d}; & l \leq d \\
    \frac{2}{\pi} \left(\arccos\left(\frac{d}{l}\right) + \frac{l - \sqrt{ l^2 - d^2}}{d} \right); & l>d.
   \end{cases}
$$

%%%%%%%%%%%%%%%%%%%%%%%%%%%%%%%%%%%%uredi cm-je!!!!!
\section{Primeri}
\begin{zgled}
Naj bo $l = 1$~cm in $d = 2$~cm. Ker je $l < d$, računamo
$$ P(A) = \frac{2 \cdot 1}{\pi \cdot 2} = \frac{1}{\pi} = 0,318. $$
\end{zgled}

\begin{zgled}
Naj bo $l = 2$~cm in $d = 1$~cm. Tokrat podatka vstavimo v~\eqref{eqn:daljsa} in dobimo
$$ P(A) = \frac{2}{\pi}\left(\arccos\left(\frac{1}{2}\right) + \frac{2 - \sqrt{3}}{1}\right) = 0,8372.$$
\end{zgled} 


%%%%%%%%%%%%%%%%%%%%%%%%%%%%%%%%%%%     LITERATURA
\bibliographystyle{plain}
\bibliography{literatura}

\end{document}