\documentclass{article}
\usepackage[utf8]{inputenc}

\usepackage{amsmath}
\usepackage{amssymb}
\usepackage{mathpartir}
\usepackage{stmaryrd}
\usepackage{newtxmath}
\usepackage{ifthen}
\usepackage{xcolor}
\usepackage{epigraph}
\usepackage{tikz-cd}
\usepackage{float}
\usepackage{framed}
\usepackage{caption}
\usepackage{tcolorbox}
\usepackage{enumitem}

\renewcommand{\and}{\\}
\newcommand{\todo}[1]{{\color{red}{#1}}}
\newcommand{\nondet}{\oplus}
\newcommand{\rel}{\sim}
\newcommand{\ctx}{\Gamma}
\newcommand{\oftype}{\;:\;}
\newcommand{\tyA}{A}
\newcommand{\tyB}{B}
\newcommand{\ctyC}{C}
\newcommand{\ctyD}{D}
\newcommand{\val}{v}
\newcommand{\judgement}[3][\ctx]{#1 \vdash #2 \oftype #3}

\title{LaTeX in Visual Studio Code}
\author{Žiga Lukšič}

%===============================================================================
\begin{document}

\maketitle

\section{Nekaj brezveznega teksta.}

Kot so opazili že mnogi~\cite{DBLP:journals/entcs/Pretnar15,DBLP:journals/jlp/BauerP15}, je res nadležno pisati naključno besedilo za račualniške delavnice. Zato bom na tej toćki raje vstavil nekaj preprostih enačb, da malo zapolnim prostor:

\[
  \judgement{v}{\tyA} = \judgement[\ctx]{v}{\tyA}
\]

\section{Še več brezveznega teksta}

\begin{tabular}{c|c|c|c|c|c}
  1 & 0 & 0 & 0 & 0 & 0 \\
  1 & 1 & 0 & 0 & 0 & 0 \\
  1 & 0 & 1 & 0 & 0 & 0 \\
  1 & 0 & 0 & 1 & 0 & 0 \\
  1 & 0 & 0 & 1 & 1 & 0 \\
  1 & 0 & 0 & 1 & 0 & 1 \\
\end{tabular}

Tukaj je naštetega nekaj sadja.

\begin{itemize}
  \item Papaja
  \item Liči
  \item Lubenica
  \item Hruška
  \item Ohrovt \todo{(???)}
  \item Jabolko
\end{itemize}

Morda bi lahko na neki točki dodali še nekaj zelenjave. Za vsak slučaj pač.

% V splošnem ne uporabljajte \newpage če niste skoraj končli s projektom
\newpage

\subsection{Se to mar nikoli ne konča?}

Če želim narediti stvari vsaj za silo prepričljive, moram pač tipkati brezvezen tekst. Sedaj bo sledila velika figura.

\begin{figure}[h]
  \label{fig:velika-figura}
  \[
	\begin{array}{rrll}
		\text{(value) type}~\tyA, \tyB
		 & ::=   & \mathtt{()}          & \text{unit type}\\
		 & \;|\; & \mathtt{bool}        & \text{boolean type}\\
		 & \;|\; & \tyA \to \ctyC     & \text{function type}\\
		 & \;|\; & \ctyC \to \ctyD   & \text{handler type}\\
		\\
		\text{computation type}~\ctyC, \ctyD
		 & ::=   & {\tyA ! \Delta \backslash \mathcal{E}}\\
		\\
		\text{signature}~\Delta
		 & ::=   & \emptyset \;|\; \Delta \cup \{{\mathit{op} \oftype \tyA \to \tyB}\}\\
		\\
		\text{value context}~\ctx
		 & ::=   & \varepsilon \;|\; \ctx, x \oftype \tyA \\
		\\
		\text{template context}~Z
     & ::=   & \varepsilon \;|\; Z, z \oftype \tyA \to * \\
     \\
		\text{computation type}~\ctyC, \ctyD
		 & ::=   & {\tyA ! \Delta \backslash \mathcal{E}}\\
		\\
		\text{signature}~\Delta
		 & ::=   & \emptyset \;|\; \Delta \cup \{{\mathit{op} \oftype \tyA \to \tyB}\}\\
		\\
		\text{value context}~\ctx
		 & ::=   & \varepsilon \;|\; \ctx, x \oftype \tyA \\
		\\
		\text{template context}~Z
		 & ::=   & \varepsilon \;|\; Z, z \oftype \tyA \to * \\
	\end{array}
\]

\end{figure}

\renewcommand\refname{Literatura}
\bibliographystyle{abbrv}
\bibliography{bibliography}

\end{document}
